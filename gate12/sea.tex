\documentclass[12pt]{article}
\usepackage{amsmath}
\usepackage{amssymb}  % for \checkmark and \times
\usepackage{graphicx}
\usepackage{geometry}
\geometry{margin=1in}

\begin{document}
\begin{figure}[h!]                                        \begin{minipage}{0.45\textwidth}  % Set the width of the image
        \includegraphics[width=\textwidth]{sun.png}  % Replace 'image.png' with your image file name
    \end{minipage} \hfill
    \begin{minipage}{0.45\textwidth}  % Set the width of the text block
        \textbf{Name : K.KARTHIK} \\
    \textbf{Batch : cometfwc026} \\
  \textbf{Date :15 may 2025}                              \end{minipage}
\end{figure}
\section*{Q.12}

\noindent
\textbf{Question:} For the output \textbf{F} to be 1 in the logic circuit shown, the input combination should be:

\vspace{1em}

\begin{center}
\includegraphics[width=0.6\textwidth]{can.png}
\end{center}

\vspace{1em}

\noindent
\textbf{Options:}

\vspace{0.5em}

\begin{tabular}{ll}
(A) A = 1, B = 1, C = 0 & (C) A = 0, B = 1, C = 0 \\
(B) A = 1, B = 0, C = 0 & (D) A = 0, B = 0, C = 1 \\
\end{tabular}

\section*{Detailed Solution}

We analyze the circuit step by step:

\begin{align*}
X &= A \oplus B \quad \text{(XOR gate output)} \\
Y &= (A \oplus B)' \quad \text{(XNOR gate output)} \\
Z &= X + Y \quad \text{(OR gate output)} \\
F &= Z \oplus C \quad \text{(Final XOR with input C)}
\end{align*}

\subsection*{Option (A): A = 1, B = 1, C = 0}
\begin{align*}
X &= 1 \oplus 1 = 0 \\
Y &= (1 \oplus 1)' = 1 \\
Z &= 0 + 1 = 1 \\
F &= 1 \oplus 0 = 1 \quad \checkmark
\end{align*}

\subsection*{Option (B): A = 1, B = 0, C = 0}
\begin{align*}
X &= 1 \oplus 0 = 1 \\
Y &= (1 \oplus 0)' = 0 \\
Z &= 1 + 0 = 1 \\
F &= 1 \oplus 0 = 1 \quad \checkmark
\end{align*}

\subsection*{Option (C): A = 0, B = 1, C = 0}
\begin{align*}
X &= 0 \oplus 1 = 1 \\
Y &= (0 \oplus 1)' = 0 \\
Z &= 1 + 0 = 1 \\
F &= 1 \oplus 0 = 1 \quad \checkmark
\end{align*}

\subsection*{Option (D): A = 0, B = 0, C = 1}
\begin{align*}
X &= 0 \oplus 0 = 0 \\
Y &= (0 \oplus 0)' = 1 \\
Z &= 0 + 1 = 1 \\
F &= 1 \oplus 1 = 0 \quad \times
\end{align*}

\section*{Conclusion}
Options (A), (B), and (C) all result in \( F = 1 \). \\

\textbf{Correct Answers: (A), (B), and (C)}

\end{document}
