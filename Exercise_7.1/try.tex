\documentclass[a4paper,12pt]{article}
\usepackage{amsmath}
\usepackage{graphicx}
\usepackage{geometry}
\usepackage{enumitem}
\usepackage{color}
\usepackage{caption}
\geometry{margin=1in}

\begin{document}
\begin{figure}[h!]
    \begin{minipage}{0.45\textwidth}  % Set the width of the image
        \includegraphics[width=\textwidth]{sun.png}  % Replace 'image.png' with your image file name
    \end{minipage} \hfill
    \begin{minipage}{0.45\textwidth}  % Set the width of the text block
        \textbf{Name : K.KARTHIK} \\
    \textbf{Batch : cometfwc026} \\
  \textbf{Date :15 may 2025}
    \end{minipage}
\end{figure}
\section*{\centering \textcolor{cyan}{EXERCISE 7.1}}

\begin{enumerate}
    \item Find the distance between the following pairs of points:
    \begin{itemize}
        \item[(i)] (2, 3), (4, 1)
        \item[(ii)] (–5, 7), (–1, 3)
        \item[(iii)] $(a, b), (-a, -b)$
    \end{itemize}

    \item Find the distance between the points (0, 0) and (36, 15). Can you now find the distance between the two towns A and B discussed in Section 7.2?

    \item Determine if the points (1, 5), (2, 3) and (–2, –11) are collinear.

    \item Check whether (5, –2), (6, 4) and (7, –2) are the vertices of an isosceles triangle.

    \item In a classroom, 4 friends are seated at the points A, B, C and D as shown in Fig. 7.8. Champa and Chameli walk into the class and after observing for a few minutes Champa asks Chameli, “Don’t you think ABCD is a square?” Chameli disagrees. Using distance formula, find which of them is correct.

    \begin{center}
        \includegraphics[width=0.7\textwidth]{cat.png} % Replace this name with the correct filename of your diagram
        \captionof{figure}{Fig. 7.8}
    \end{center}

    \item Name the type of quadrilateral formed, if any, by the following points, and give reasons for your answer:
    \begin{itemize}
        \item[(i)] (–1, 2), (1, 0), (–1, –2), (–3, 0)
        \item[(ii)] (–3, 5), (3, 1), (0, 3), (–1, –4)
        \item[(iii)] (4, 5), (7, 6), (4, 3), (1, 2)
    \end{itemize}

    \item Find the point on the $x$-axis which is equidistant from (2, –5) and (–2, 9).

    \item Find the values of $y$ for which the distance between the points $P(2, -3)$ and $Q(10, y)$ is 10 units.

    \item If $Q(0, 1)$ is equidistant from $P(5, –3)$ and $R(x, 6)$, find the values of $x$. Also find the distances $QR$ and $PR$.

    \item Find a relation between $x$ and $y$ such that the point $(x, y)$ is equidistant from the point (3, 6) and (–3, 4).
\end{enumerate}

\vfill
\hfill \textit{Mathematics} \hfill \textbf{162}

\end{document}




